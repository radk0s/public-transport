
\documentclass[a4paper]{article}

\usepackage[utf8]{inputenc}
\usepackage{polski}
\usepackage[polish]{babel}
\usepackage{url}
\usepackage{hyperref}
\usepackage{pgffor}

\def\sectionautorefname{rozdziale}

\usepackage{natbib}
\usepackage{keyval}
\usepackage{usebib}
\bibliographystyle{abbrv}
\bibinput{bibliografia}

\usepackage{listings}
\lstset{tabsize=4, frame=tb, numbers=none}

\usepackage{tabulary}

\usepackage{pdfpages}

\graphicspath{{images/}}

\author{Dawid Szmigielski, Radosław Chamot, Michał Liszcz}

% TODO zmienic tytul
\title{Stochastyczne algorytmy obliczeniowe \\ trasy}

% \supervisor{prof. dr hab. inż. Robert Schaefer}

\date{2015}

\begin{document}

\maketitle
\newpage

\tableofcontents
\newpage


%% section 1 %%%%%%%%%%%%%%%%%%%%%%%%%%%%%%%%%%%%%%%%%%%%%%%%%%%%%%%%%%%%%%%%%%

\section{Wprowadzenie} \label{sec:intro}
Problem optymalizacji sieci transportu miejskiego jest tematem licznych badań. Zapotrzebowanie na transport publiczny stale rośnie. Przewoźnicy dążą do minimalizacji kosztów operacyjnych a jednocześnie do zapewnienia pasażerom najwyższej jakości usług (minimalny czas oczekiwania na przystankach, duża częstotliwość kursowania na zatłoczonych liniach czy optymalne zapełnienie pojazdów). W tej pracy przedstawiamy własną propozycję rozwiązania problemu przydziału pojazdów komunikacji miejskiej do wyznaczonych tras.

\subsection{Opis problemu}
Poszukujemy optymalnego przydziału pojadzów komunikacji miejskiej do ustalonych tras linii. Linie łączą przystanki. Na odcinkach między przystankami panuje stałe natężenie ruchu pasażerów. Ścisłe sformułowanie problemu znajduje się w \autoref{sec:solution}.



%% section 2 %%%%%%%%%%%%%%%%%%%%%%%%%%%%%%%%%%%%%%%%%%%%%%%%%%%%%%%%%%%%%%%%%%

\section{Przegląd rozwiązań}
Temat optymalizacji lini transportu miejskiego był poruszany w wielu pracach. Zaproponowano wiele różnych modeli, z różnymi ograniczeniami i różnymi kryteriami optymalizacji. Przedstawiamy tutaj najważniejsze spośród rozwiązań wykorzystujących algorytmy stochastyczne.



%% section 3 %%%%%%%%%%%%%%%%%%%%%%%%%%%%%%%%%%%%%%%%%%%%%%%%%%%%%%%%%%%%%%%%%%

\section{Opis algorytmu} \label{sec:solution}
Blah blah blah. \cite{bib-route-optimization}



%% section 4 %%%%%%%%%%%%%%%%%%%%%%%%%%%%%%%%%%%%%%%%%%%%%%%%%%%%%%%%%%%%%%%%%%

\section{Wyniki}
Blah blah.


\nocite{bib-transit-oriented}

\bibliography{bibliografia}
\newpage

\listoffigures
\newpage

\end{document}
